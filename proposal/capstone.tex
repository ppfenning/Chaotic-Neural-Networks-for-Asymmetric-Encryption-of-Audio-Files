\documentclass[conference]{IEEEtran}
\IEEEoverridecommandlockouts
% The preceding line is only needed to identify funding in the first footnote. If that is unneeded, please comment it out.
\usepackage{cite}
\usepackage{amsmath,amssymb,amsfonts}
\usepackage{algorithmic}
\usepackage{graphicx}
\usepackage{textcomp}
\usepackage{stmaryrd}
\newcommand{\BibTeX}{\textrm{B \kern -.05em \textsc{i \kern -.025em b} \kern -.08em
T \kern -.1667em \lower .7ex \hbox{E} \kern -.125emX}}
\begin{document}

\title{The Efficacy of Chaotic Neural Networks for Asymmetric Encryption of Audio Files}

\author{\IEEEauthorblockN{1\textsuperscript{st} Patrick Pfenning}
\IEEEauthorblockA{\textit{School of Computing and Data Science} \\
\textit{Wentworth Institute of Technology}\\
Boston, MA \\
pfenningp@wit.edu}
}

\maketitle

\begin{abstract}
This paper focuses on developing a new algorithm for encrypting audio files using a Chaotic Neural Network (CNN)
Said algorithm is benchmarked for its efficiency and effectiveness against two standard encryption methods, Advanced Encryption Standard (AES) and Rivest–Shamir–Adleman (RSA).
The intent of each layer of the network utilizes chaotic maps, which can be used to generate pseudo-random arrays.
These arrays are transformed into bitmaps which are used to encrypt files using Waveform Audio File Format.
The results show that the proposed algorithm is more efficient and effective than both AES and RSA, but may be easier to crack.
\end{abstract}

\begin{IEEEkeywords}
chaos, encryption, audio, neural networks, pseudo-random
\end{IEEEkeywords}

\section{Introduction}\label{sec:introduction}
The world we live in has become wildly dependent on the fast transfer of information.
Everything, from our work and education to our entertainment and social life, often requires the internet as a medium.
With this abundance of information, how can one ensure that their personal data remains private to bad actors?
Enter the field of encryption, the process of encoding information.
Encryption allows one to take a plaintext message we wish to transmit and transform it into a ciphertext.
This ciphertext is illegible to those who do not have a \textit{key}.

Most modern encryption processes create a \textit{psuedo-random} key via a defined algorithm.
These algorithms can be broken into two schemes: \textbf{Symmetric-key Encryption} and \textbf{Public-key Encryption.}
Symmetric-key Encryption algorithms create a single secret key which both the sender and receiver have.
This key is used to both encrypt and decrypt the message.
The security of this method directly depends on the holders of the key as anyone who has it can read the cipher.
Public-key Encryption algorithms require the message receiver to generate both a public and private key.
The receiver shares the public key with the trusted sender who uses it to encrypt a message.
The message is sent to the receiver where the private key is used to decrypt it.
Because the private key is the only way to decipher the message, public-keys can be shared with impunity.
Though both methods do nothing to prevent the cipher from being intercepted by a third party, if the algorithm is strong enough, it should be illegible.

Billions of packets of information flow throughout the internet on a daily basis.
The larger the packet, the longer the encryption will take.
In this paper, we will develop our own algorithm to encrypt several audio files of varying sizes.
The aforementioned algorithm will be developed using what is known as a \textbf{Chaotic Neural Network} (CNN).
This algorithm will be benchmarked for both efficiency and effectiveness against one symmetric-key method, \textbf{Advanced Encryption Standard} (AES), as well as one public-key method, the \textbf{Rivest–Shamir–Adleman} (RSA).

\section{Literature Review}\label{sec:literature-review}
\input{literature-review}

\section{Methodology}\label{sec:methodology}

\subsection{Generalization}\label{subsec:generalization}

Suppose there exists some file $A$ which we wish to encrypt using a CNN\@.
The first layer of our network is to transform $A \rightarrow A^\prime$, such that $A^\prime$ is the matrix containing the header bytes of the audio file $A$~\cite{app112110190}.
We now define the integer $N$ as the byte length of $A^\prime$.
Each subsequent layer of our network will consist of $N$ nodes~\cite{Ahmad2016}, with weights generated by the chosen chaotic map.
Each map will have the set of initial conditions corresponding to:

\begin{enumerate}
    \item The chosen parameters of the map.
    \item Input variable for each dimension of the map.
    \item Some integer $k$ defining the number of iterations to remove from the generator.
\end{enumerate}

We then plug in our parameters, supply our input variables and iterate the map $k+N$ times.
Once removing the first $k$ values of the list, we are left with a matrix $Z_{d\times N}$ such that $d$ is the dimensionality of our map.
We then apply the following transformation~\cite{Lokesh,app112110190} to our matrix:

\begin{equation}\label{eq:bitmap}
    W_i = \bigoplus\big\lfloor\left{|} Z_i \right{|} \times 10^{10} \big\rfloor \mod 256
\end{equation}

Such that $i$ denotes which hidden layer of our network we are on.
$W_i$ is the logical XOR to the byte-wise representation of our transformed map outputs.
Each hidden layer of our network becomes a column of our overall weighted matrix $W$ such that

\begin{equation*}\label{eq:weights}
    W = \big[W_1 | \ldots | W_n \big]
\end{equation*}

We then XOR  $A^\prime$ with all columns of $W$ to obtain our the encrypted header $E^\prime$, and in turn the encrypted audio file $E$.
The matrix $W$ acts as both our encryption and decryption key because

\begin{equation*}
    A^\prime = A^\prime \oplus W \oplus W = E^\prime \oplus W\\
\end{equation*}

Theoretically, this methodology should work for any set of chaotic maps we choose.
Through the rest of this paper we will develop an algorithm using this methodology with a set of maps.

\subsection{Choosing Our Maps}\label{subsec:choosing-our-maps}

We have chosen audio as our medium for encryption.
For ease of use, we will be using Waveform Audio File Format.
Byte representations of these files are one dimensional and thus fit our methodology.
Due to the one dimensional nature, we may use any size dimensional map to fit our needs.
We have chosen to develop a network consisting of four hidden layers, each representing their own map.

The first layer we will use is the Hénon Map\cite{Hamdy}

\begin{align}\label{eq:Henon}
    x_{n+1} &= 1 - ax_n^2 + y_n \\
    y_{n+1} &= bx_n
\end{align}

This is a 1-D map that will act as a \textit{diffusion} layer in our system.
Chaotic behavior occurs here with parameters $a=0.3$ and $b=1.4$.
Our second layer will be the Ikeda Map~\cite{app112110190}

\begin{align}\label{eq:Ikeda}
    x_{n+1} &= 1 + u(x_n \cos(t_n) - y_n \sin(t_n))\\
    y_{n+1} &= u(x_n \sin(t_n) + y_n \cos(t_n))\\
    t_{n+1} &= \beta - \frac{\gamma}{1+x_{n+1}^2+y_{n+1}^2}
\end{align}

Where this system develops a chaotic attractor whenever $u\ge0.6$.
Adding complexity, we turn to the 3-D map known as the Lorenz attractor~\cite{Naik2022}.

\begin{align}\label{eq:Lorenz}
    \frac{dx}{dt} &= \sigma (y - x)\\
    \frac{dy}{dt} &= x (\rho - z) - y\\
    \frac{dz}{dt} &= xy - \beta z
\end{align}

We will use parameter values $\sigma=10$, $\beta=\frac{8}{3}$ and $\rho=28$.
Finally, we will use the most common chaotic function, the Logistic Map.

\begin{equation}\label{eq:Logistic}
    x_{n+1} = r x_n (1 - x_n)
\end{equation}

Such that $r=4$.
It should be noted that $x_0 \in [0,1]$ for chaotic behavior to occur.

\section{Algorithm}\label{sec:algorithm}

\subsection{Key Generation}\label{subsec:key-generation}

\subsection{Diffusion}\label{subsec:diffusion}

\subsection{Encryption}\label{subsec:encryption}

\subsection{Decryption}\label{subsec:decryption}

\subsection{De-Diffusion}\label{subsec:de-diffusion}

\section{Experimentation and Results}\label{sec:experimentation-and-results}

Here we test our algorithm against several existing methods

\subsection{Processing Time and Complexity}\label{subsec:processing-time-and-complexity}

\subsection{Histogram Analysis}\label{subsec:histogram-analysis}

\subsection{Correlation Analysis}\label{subsec:correlation-analysis}

\subsection{Peak Signal to Noise Ratio}\label{subsec:peak-signal-to-noise-ratio}

\subsection{Encryption Quality}\label{subsec:encryption-quality}

\subsection{Vulnerability to Attacks}\label{subsec:vulnerability-to-attacks}

\subsection{Key Sensitivity}\label{subsec:key-sensitivity}

\section{Future Work}\label{sec:future-work}

\section{Conclusion}\label{sec:conclusion}

\section{Acknowledgments}\label{sec:acknowledgments}

\bibliographystyle{ieeetr}
\bibliography{bib/bibliography}

\end{document}
