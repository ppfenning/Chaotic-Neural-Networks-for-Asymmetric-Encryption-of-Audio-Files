\documentclass[conference]{IEEEtran}
\IEEEoverridecommandlockouts
% The preceding line is only needed to identify funding in the first footnote. If that is unneeded, please comment it out.
\usepackage{cite}
\usepackage{amsmath,amssymb,amsfonts}
\usepackage{algorithmic}
\usepackage{graphicx}
\usepackage{textcomp}
\newcommand{\BibTeX}{\textrm{B \kern -.05em \textsc{i \kern -.025em b} \kern -.08em
T \kern -.1667em \lower .7ex \hbox{E} \kern -.125emX}}
\begin{document}

\title{The Efficacy of Chaotic Neural Networks for Asymmetric Encryption of Audio Files}

\author{\IEEEauthorblockN{1\textsuperscript{st} Patrick Pfenning}
\IEEEauthorblockA{\textit{School of Computing and Data Science} \\
\textit{Wentworth Institute of Technology}\\
Boston, MA \\
pfenningp@wit.edu}
}

\maketitle

\begin{abstract}
This document is a model and instructions for \LaTeX.
This and the IEEEtran.cls file define the components of your paper [title, text, heads, etc.]. *CRITICAL: Do Not Use Symbols, Special Characters, Footnotes,
or Math in Paper Title or Abstract.
\end{abstract}

\begin{IEEEkeywords}
component, formatting, style, styling, insert
\end{IEEEkeywords}

\section{Introduction}\label{sec:introduction}

The world we live in has become wildly dependent on fast transfer of information.
Everything from our work and education to our entertainment and social life often requires the internet as a medium.
With this abundance of information, how can one ensure that their personal data remains private to bad actors?
Enter the field encryption, the process of encoding information.
Encryption allows one to take a plaintext message we wish to transmit and transforms it into a ciphertext.
Said ciphertext is illegible to those who do not have a \textit{key}.

Most modern encryption processes create a \textit{psuedo-random} key via a defined algorithm.
These algorithms can be broken into two schemes: \textbf{Symmetric-key Encryption} and \textbf{Public-key Encryption.}
Symmetric-key Encryption algorithms create a single secret-key which both the sender and receiver have.
This key is used to both encrypt and decrypt the message.
The security of this method directly depends on the holders of the key as anyone who has it can read the cipher.
Public-key Encryption algorithms require the message receiver to generate both a public and private key.
The receiver shares the public key with the trusted sender who uses it to encrypt a message.
Said message is sent to the receiver where the private key is used to decrypt it.
Because the private key is the only way to decipher the message, public-keys can be shared with impunity.
Though both methods do nothing to prevent the cipher from being intercepted by a third party, if the algorithm is strong enough, it should be illegible.

Billions of packets of information flow throughout the internet on a daily basis.
The larger the packet, the longer the encryption will.
In this paper we will develop our own algorithm to encrypt several audio files of varying.
The aforementioned algorithm will be developed using what is known as a \textbf{Chaotic Neural Network.}
This algorithm will be benchmarked for both efficiency and effectiveness against one symmetric-key method, \textbf{Advanced Encryption Standard} (AES), as well as one public-key method, the \textbf{Rivest–Shamir–Adleman} algorithm (RSA).


\section{Background}\label{sec:background}

There are many possible algorithms for encryption.
One trait they share is the creation of a large \textit{psuedo-random} sequence used to create a cipher.
Chaos is fast and efficient way to create these sequences.

\subsection{What is Chaos?}\label{subsec:what-is-chaos?}

Chaos is the scientific study of dynamic systems.
These systems are abundant in nature and are highly sensitive to initial conditions.
Edward Lorenz, the founding-father of chaos theory, once described chaos as

\begin{framed}
    \begin{quote}
        When the present determines the future, but the approximate present does not approximately determine the future.~\cite{Alligood}
    \end{quote}
\end{framed}


This

\section{Literature Review}\label{sec:literature-review}

\subsection{Chaotic neural network based pseudo-random sequence generator for cryptographic applications}\cite{Lokesh}
\subsection{Cryptography using Artificial Neural Networks}\cite{Gujral}
\subsection{Public key Cryptography Based on Chaotic Neural Network}\cite{Hamdy}
\subsection{Chaos: An Introduction to Dynamical Systems}\cite{Alligood}
\subsection{A Novel Color Image Encryption Scheme Based on Hyperchaos and Hopfield Chaotic Neural Network}\cite{Wu2022}
\subsection{An Image Encryption Algorithm Based on a New Fractional Order Chaotic Neural Network}\cite{Li2022}
\subsection{Comparison of Cryptography by Chaotic Neural Network and by AES}\cite{Skovajsova2019}
\subsection{Design of chaotic neural network based method for cryptographic substitution box}\cite{Ahmad2016}
\subsection{The Study of Chaotic Neural Network and its Applications in Associative Memory}\cite{He1999}
\subsection{On Chaotic Neural Network Design: A New Framework}\cite{Qin2017}
\subsection{An Empirical Investigation of Using ANN Based N-State Sequential Machine and Chaotic Neural Network in the Field of Cryptography}\cite{Shukla2012}
\subsection{A Review on Applications of Chaotic Maps in Pseudo-Random Number Generators and Encryption}\cite{Naik2022}

\section{Building the Network}\label{sec:building-the-network}

\subsection{Chaotic Functions}\label{subsec:chaotic-functions}

\begin{itemize}
    \item Hénon map
    \item Logistic map
    \item Lorenz system
    \item Tent map
    \item Horseshoe map
    \item \ldots
\end{itemize}

\subsection{Network and Diagram}\label{subsec:network-and-diagram}

Outline inputs and hidden layers here

\begin{figure}[!ht]
    \centering
    \includegraphics{figures/CNN}
    \caption{Example CNN Architecture}
    \label{fig:CNN}
\end{figure}

\section{Algorithm}\label{sec:algorithm}

\subsection{Key Generation}\label{subsec:key-generation}

\subsection{Diffusion}\label{subsec:diffusion}

\subsection{Encryption}\label{subsec:encryption}

\subsection{Decryption}\label{subsec:decryption}

\subsection{De-Diffusion}\label{subsec:de-diffusion}

\section{Experimentation and Results}\label{sec:experimentation-and-results}

Here we test our algorithm against several existing methods

\subsection{Processing Time and Complexity}\label{subsec:processing-time-and-complexity}

\subsection{Histogram Analysis}\label{subsec:histogram-analysis}

\subsection{Correlation Analysis}\label{subsec:correlation-analysis}

\subsection{Peak Signal to Noise Ratio}\label{subsec:peak-signal-to-noise-ratio}

\subsection{Encryption Quality}\label{subsec:encryption-quality}

\subsection{Vulnerability to Attacks}\label{subsec:vulnerability-to-attacks}

\subsection{Key Sensitivity}\label{subsec:key-sensitivity}

\section{Future Work}\label{sec:future-work}

\section{Conclusion}\label{sec:conclusion}

\section{Acknowledgments}\label{sec:acknowledgments}

\bibliographystyle{ieeetr}
\bibliography{bib/bibliography}

\end{document}
